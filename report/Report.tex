\documentclass{article}
\usepackage[utf8]{inputenc}
\usepackage{graphicx}
\usepackage{epstopdf}
\usepackage{caption}
\usepackage{subcaption}
\usepackage{multirow}
\usepackage{hyperref}
\usepackage{url}
\usepackage{seqsplit}
\hypersetup{pdfstartview={FitH null null null}}
\usepackage{amssymb,amsmath}
\usepackage{amsthm}
\usepackage{empheq}
\usepackage{algorithm,algpseudocode}
\usepackage[margin=1.5in]{geometry}
\usepackage{listings}
\usepackage{program}
\lstset{language=Python} 

\usepackage{listings}
\usepackage{color} %red, green, blue, yellow, cyan, magenta, black, white
\definecolor{mygreen}{RGB}{28,172,0} % color values Red, Green, Blue
\definecolor{mylilas}{RGB}{170,55,241}


\title{Stochastic optimization of an objective function to build a 3D chromosomal model}
\author{Caiwei Wang, Xiaokai Qian, Sean Lander, \\Haipei Fan, Puneet Gaddam, Brett Koonce\\\\University of Missouri - Columbia}

\date{May 7, 2014}

\algloopdefx{NoEndIf}[1]{\textbf{If} #1 \textbf{then}}

\begin{document}

\maketitle

\section{Abstract}

First, we construct a congugate gradient (CG) descent searcher to build a 3D model of a human chromosome using the method advanced by Trieu and Cheng in a recent paper.  Next, we utilize two different stochastic methods, simulated annealing (SA) and Markov-Chain Monte-Carlo (MCMC), to duplicate our CG results.  Then, we build a PDB model of the final chromosome and visualize the process of solving the objective function.  Ultimately, we compare our results with other research in this domain, discuss difficulties we encountered and potential future improvements that could be made.

\section{Introduction}




\section{Citations}

We thank the following tools and papers: \\

Tuan Trieu and Jianlin Cheng.  Large-scale reconstruction of 3D structures of human chromosomes from chromosomal contact data.  Nucl. Acids Res. first published online January 24, 2014. doi:10.1093/nar/gkt1411


\section{Visualization}



\end{document}
